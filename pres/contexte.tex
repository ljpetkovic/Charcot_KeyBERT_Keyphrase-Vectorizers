\begin{frame}{Définition de la tâche}
Angl. \textit{keyphrases} : \og{}phrases-clés\fg{}
\begin{itemize}
\item séquences de plusieurs mots (ex. \textit{sclérose latérale amyotrophique})
\item reflètent plus précisément le contexte sémantique du texte \\\small{$\neq$ mots-clés, ex. \textit{sclérose}}
\end{itemize}
\bigskip
\begin{table}[h!]
\justifying
\begin{tabular}{ll}
Extraction &
  Prédiction \\
  \small
\begin{tabular}[l]{@{}l@{}}
 Processus de \underline{sélection} d'un ensemble\\ de phrases les plus pertinentes à partir \\d'un texte donné \citep{schopf2022}.\end{tabular} &
 \small
  \begin{tabular}[l]{@{}l@{}}
  Processus de \underline{génération} des phra-\\ses-clés qui résument parfaitement\\ un document donné \citep{xie2023}.\end{tabular}
\end{tabular}
\end{table}
%Extraction de phrases-clés
%
%Processus de \underline{sélection} automatique d'un petit ensemble de phrases les plus pertinentes à partir d'un texte donné \citep{schopf2022}.
%
%\begin{block}{Prédiction de phrases-clés}
%\justifying
%Processus de \underline{génération} des phrases-clés qui résument parfaitement un document donné \citep{xie2023}.
%\end{block} 
\end{frame}
